\documentclass[stu,12pt,floatsintext]{apa7}
% Document class input explanation ________________
% LaTeX files need to start with the document class, so it knows what it's using
% - This file is using the apa7 document class, as it has a lot of the formatting built in
% There are two sets of brackets in LaTeX, for each command (the things that start with the slash \ )
% - The squiggle brackets {} are mandatory for executing the command
% - The square brackets [] are options for that command. There can be more than one set of square brackets for some commands
% Options used in this document (general note - for each of these, if you want to use the other options, swap it out in that spot in the square brackets):
% - stu: this sets the `document mode' as the "student paper" version. Other options are jou (journal), man (manuscript, for journal submission), and doc (a plain document)
% --- The student setting includes things like 'duedate', 'course', and 'professor' on the title page. If these aren't wanted/needed, use the 'man' setting. It also defaults to including the tables and figures at the end of the document. This can be changed by including the 'floatsintext' option, as I have for you. If the instructor wants those at the end, remove that from the square brackets.
% --- The manuscript setting is roughly what you would use to submit to a journal, so uses 'date' instead of 'duedate', and doesn't include the 'course' or 'professor' info. As with 'stu', it defaults to putting the tables and figures at the end rather than in text. The same option will bump those images in text.
% --- Journal ('jou') outputs something similar to a common journal format - double columned text and figurs in place. This can be fun, especially if you are sumbitting this as a writing sample in applications.
% --- Document ('doc') outputs single columned, single spaced text with figures in place. Another option for producing a more polished looking document as a writing sample.
% - 12pt: sets the font size to 12pt. Other options are 10pt or 11pt
% - floatsintext: makes it so tables and figures will appear in text rather than at the end. Unforunately, not having this option set breaks the whole document, and I haven't been able to figure out why. IT's GREAT WHEN THINGS WORK LIKE THEY'RE SUPPOSED TO.

\usepackage[american]{babel}
\usepackage{graphicx}
\usepackage{amsmath}
\usepackage{amssymb}
\usepackage{float}
\usepackage{tikz}
\usepackage{circuitikz}
\usepackage{pgfplots}
\usepackage{pgfplotstable}
\usepackage{siunitx}
\usepackage{booktabs}
\usepackage{listings}
\usepackage{subcaption}

\usepackage{csquotes} % One of the things you learn about LaTeX is at some level, it's like magic. The references weren't printing as they should without this line, and the guy who wrote the package included it, so here it is. Because LaTeX reasons.
\usepackage[style=apa,sortcites=true,sorting=nyt,backend=biber]{biblatex}
% biblatex: loads the package that will handle the bibliographic info. Other option is natbib, which allows for more customization
% - style=apa: sets the reference format to use apa (albeit the 6th edition)
\DeclareLanguageMapping{american}{american-apa} % Gotta make sure we're patriotic up in here. Seriously, though, there can be local variants to how citations are handled, this sets it to the American idiosyncrasies 
\addbibresource{bibliography.bib} % This is the companion file to the main one you're writing. It contains all of the bibliographic info for your references. It's a little bit of a pain to get used to, but once you do, it's the best. Especially if you recycle references between papers. You only have to get the pieces in the holes once.`

\usepackage[T1]{fontenc} 
\usepackage{mathptmx} % This is the Times New Roman font, which was the norm back in my day. If you'd like to use a different font, the options are laid out here: https://www.overleaf.com/learn/latex/Font_typefaces
\usepackage{xstring}
% Alternately, you can comment out or delete these two commands and just use the Overleaf default font. So many choices!

\DeclareCaptionFormat{custom}{\textbf{#1#2}\textit{\small #3}}
\captionsetup{format=custom}
\pgfplotsset{compat=1.18}

% Title page stuff _____________________
\title{Lab #: Title } % The big, long version of the title for the title page
\shorttitle{Lab #} % The short title for the header
\author{Name}
\duedate{Date}
\affiliation{University of California \- Santa Cruz, Jack Baskin School of Engineering}
\course{Course Name}
\professor{}

\begin{document}

\newcommand{\dataslewplot}[5]{%
  \def\firstString{#2}
  \def\secondString{#3}
  \def\xmax{1000}
  \def\xmin{0.1}
  \def\width{14cm}
  \def\height{7cm}

  \begin{figure}[H]
    \centering
    \caption{#1}
    \label{fig:#2_#3_slew_plot}
    \begin{tikzpicture}
      \begin{axis}[
        width=\width,
        height=\height,
        xmode=log,
        axis y line*=left,
        ymin = #4, ymax = #5,
        xmin=\xmin, xmax=\xmax,
        scale only axis,
        xlabel={Frequency (kHz)},
        ylabel={Slew Rate (V/ms)},
        log origin=infty, 
        grid=both,
      ]

        \addplot [blue, only marks] table[
          only marks,
          x expr={\thisrow{Frequency (kHz)}},
          y expr={\thisrow{Slew_Rate(V/ms)}},
          col sep=semicolon
        ] {./data/\firstString_\secondString.csv};

      \end{axis}
    \end{tikzpicture}
  \end{figure}
}

% #1 = title
% #2 = file_name in data folder
% #3 = y1 min
% #4 = y1 max
% #5 = y2 min
% #6 = y2 max

\newcommand{\databodeplot}[6]{%
  \def\xmax{1000}
  \def\xmin{0.1}
  \def\width{14cm}
  \def\height{7cm}
  \begin{figure}[H]
    \centering
    \caption{#1}
    \begin{tikzpicture}
      \label{#2}
      \pgfplotsset{set layers} 
      \begin{axis}[
        width=\width,
        height=\height,
        xmode=log,
        axis y line*=left,
        ymin = #3, ymax = #4,
        xmin=\xmin, xmax=\xmax,
        scale only axis,
        xlabel={Frequency (kHz)},
        ylabel={Voltage Gain (dB)},
        log origin=infty, 
        grid=both,
        legend to name=plotlegend1
      ]

        \addplot [blue, only marks] table[
          only marks,
          x expr={\thisrow{Frequency (kHz)}},
          y expr={20 * log10(\thisrow{V_out_pp(V)} / \thisrow{V_in_pp(V)})},
          col sep=semicolon
        ] {./data/#2.csv};
        \addlegendentry{Gain Value $\text{(dB)}$};

      \end{axis}

      \begin{axis}[
        width=\width,
        height=\height,
        xmode=log,
        ymin = #5, ymax = #6,
        xmin=\xmin, xmax=\xmax,
        scale only axis,
        ylabel={Phase (degree)},
        axis y line*=right,
        axis x line=none,
        legend to name=plotlegend2
      ]

        \addplot [red, only marks] table[
          x expr={\thisrow{Frequency (kHz)}},
          y expr={\thisrow{Phase(degree)}},
          col sep=semicolon
        ] {./data/#2.csv};
        \addlegendentry{Phase $(^{\circ})$};

      \end{axis}
      \node at (current bounding box.north) [anchor=north,xshift=50pt,yshift=-5pt] {\ref{plotlegend1}};
      \node at (current bounding box.north) [anchor=north,xshift=-50pt,yshift=-5pt] {\ref{plotlegend2}};
    \end{tikzpicture}
  \end{figure}
}

\newcommand{\image}[3]{
  \def\firstString{#1}
  \def\secondString{#2}
  \def\width{14cm}
  \def\height{7cm}
  \begin{figure}[H]
    \centering
    \caption{\firstString}
    \includegraphics[
      width=\width,
      height=\height,
      keepaspectratio=true
    ]{./images/\secondString}
    \begin{center}
      \textit{#3}
    \end{center}
  \end{figure}
}

\newcommand{\schematic}[1]{
  \input{schematics/#1.tex}
}



%\sisetup {
%  round-mode      = places,
%  round-precision = 4,
%}

\maketitle % This tells LaTeX to make the title page

\section{Overview}

Lorem ipsum odor amet, consectetuer adipiscing elit. Quisque dis quis et placerat morbi faucibus ad. Sit diam habitasse ad; lacus curabitur vivamus rutrum. Vel maximus sagittis magna arcu diam; felis sodales donec. 
\begin{align}
  S_{r} = \max \left|\frac{v(t)}{dt}\right|
\end{align}
\noindent Pharetra sociosqu quam donec orci aenean at. Laoreet vehicula volutpat ornare quam; nam tempus semper fusce. Id sodales nulla fusce ut ridiculus cursus lacus. Purus dolor ridiculus dapibus sociosqu turpis elit aliquet.

\image{Slew Example}
{slew_example.jpeg}
{Sample oscilloscope reading of resulting slew given a square input signal.}

Avolutpat lacus auctor mi himenaeos. Convallis aliquet habitant congue ad leo mollis. Libero sollicitudin facilisi nec mollis dui amet tortor montes elementum. Leo pellentesque integer sodales faucibus mi montes vitae ultricies. Purus molestie sociosqu nam aptent leo. Maximus pretium platea purus donec aliquam praesent. Eu eros aenean sit et curae cras, est libero. Ut felis laoreet at aliquam sagittis; integer lectus ultricies. Cras phasellus nisi massa non congue imperdiet conubia nam. Congue mauris molestie parturient; orci imperdiet et.
\begin{align}
  \hat{R}_{1,l} &= 0.995 \text{k}\Omega\\
  \hat{R}_{1,h} &= 9.99 \text{k}\Omega\\
  \hat{R}_{2}   &= 10.02 \text{k}\Omega
\end{align}
Primis pulvinar litora feugiat torquent curae aptent. Suscipit consequat penatibus id natoque nascetur quisque diam semper. Felis netus odio in hac inceptos ipsum integer. Pellentesque rutrum eros mollis cras pellentesque per commodo. Sociosqu pharetra amet; dolor et etiam urna. Varius diam nisi parturient imperdiet interdum augue. Finibus elit aenean sed libero ridiculus ad. Suspendisse parturient phasellus commodo in ante eleifend dis a. Scelerisque lorem porta leo cursus aliquet phasellus per.

\subsection{Some Other Section}

\schematic{negative_feedback}

Ornare neque in phasellus bibendum ad sit. Vivamus primis cras lacus varius vehicula hendrerit nulla parturient. Amet sodales lacus eleifend suspendisse sagittis augue enim. Vivamus non nam congue rutrum amet dignissim nunc aliquam. Tincidunt purus proin ligula venenatis ex; elit semper justo. Natoque in enim ullamcorper; sed orci primis class eu. Congue quis nisi ex purus quam posuere. Dignissim justo odio mi quisque litora potenti scelerisque nec posuere. Lectus pharetra commodo; diam ultrices eu ex. Urna tellus mauris potenti ex duis aliquam rutrum.

\begin{itemize}
  \item one thing
  \item two things
  \item three things
\end{itemize}

\begin{enumerate}
  \item one thing
  \item two things
  \item three things
\end{enumerate}

\end{document}

%% 
%% Copyright (C) 2019 by Daniel A. Weiss <daniel.weiss.led at gmail.com>
%% 
%% This work may be distributed and/or modified under the
%% conditions of the LaTeX Project Public License (LPPL), either
%% version 1.3c of this license or (at your option) any later
%% version.  The latest version of this license is in the file:
%% 
%% http://www.latex-project.org/lppl.txt
%% 
%% Users may freely modify these files without permission, as long as the
%% copyright line and this statement are maintained intact.
%% 
%% This work is not endorsed by, affiliated with, or probably even known
%% by, the American Psychological Association.
%% 
%% This work is "maintained" (as per LPPL maintenance status) by
%% Daniel A. Weiss.
%% 
%% This work consists of the file  apa7.dtx
%% and the derived files           apa7.ins,
%%                                 apa7.cls,
%%                                 apa7.pdf,
%%                                 README,
%%                                 APA7american.txt,
%%                                 APA7british.txt,
%%                                 APA7dutch.txt,
%%                                 APA7english.txt,
%%                                 APA7german.txt,
%%                                 APA7ngerman.txt,
%%                                 APA7greek.txt,
%%                                 APA7czech.txt,
%%                                 APA7turkish.txt,
%%                                 APA7endfloat.cfg,
%%                                 Figure1.pdf,
%%                                 shortsample.tex,
%%                                 longsample.tex, and
%%                                 bibliography.bib.
%% 
%%
%%
%% This is file `./samples/shortsample.tex',
%% generated with the docstrip utility.
%%
%% The original source files were:
%%
%% apa7.dtx  (with options: `shortsample')
%% ----------------------------------------------------------------------
%% 
%% apa7 - A LaTeX class for formatting documents in compliance with the
%% American Psychological Association's Publication Manual, 7th edition
%% 
%% Copyright (C) 2019 by Daniel A. Weiss <daniel.weiss.led at gmail.com>
%% 
%% This work may be distributed and/or modified under the
%% conditions of the LaTeX Project Public License (LPPL), either
%% version 1.3c of this license or (at your option) any later
%% version.  The latest version of this license is in the file:
%% 
%% http://www.latex-project.org/lppl.txt
%% 
%% Users may freely modify these files without permission, as long as the
%% copyright line and this statement are maintained intact.
%% 
%% This work is not endorsed by, affiliated with, or probably even known
%% by, the American Psychological Association.
%% 
%% ----------------------------------------------------------------------
